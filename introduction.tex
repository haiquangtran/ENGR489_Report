\chapter{Introduction}\label{C:intro}

With the vast food options around us, it is often difficult to decide what to eat. On some days we may feel like eating our favourite foods, but on other days we could feel more explorative, wanting to try something new. There are many factors that determine our decisions on what we want to eat such as our preferences, our current mood, and what food providers are nearby. Other factors that may contribute is the price of the food dish, the cuisine type, the meat type, and so forth. Additionally, some of us may have food intolerances which restrict our food options.

What’s On The Menu (WOTM) is a web-centric mobile application that is focused on recommending food dishes to users based on their personal preferences, previous actions such as likes/dislikes, and other factors such as location, price, and so forth. These personalised recommendations are important as it saves time for users and allows users to explore preferred food options they otherwise might not have discovered alone. 

Existing applications such as Yelp and Foursquare have similar concepts but lack personalised recommendations for specific food dishes. These applications focus on a broader scope such as popular restaurants, coffee places, activities, and so forth. This can often lead to a cluttered interface, affecting the user experience, and may provoke confusion in users. Food spotting is an application that focuses on food dishes but does not provide personalised recommendations to users. Instead, they show popular food dishes and focus on crowd-sourcing, relying on users to upload food dishes. What's On The Menu (WOTM) differs as the goal is to provide personalised food recommendations to users through collaborative filtering. 

Collaborative filtering is a popular recommendation technique used to recommend items to users based on previous rating patterns and the behaviours of others \cite{itembased, schafer2007collaborative, survey}. 

\todo{State explicit goals}
The goal of this project is to find which types of collaborative filtering algorithms best suit the WOTM application. Suitability of these algorithms will be considered on factors that are important for WOTM. The recommender system will have to account for a range of factors from the user experience, the accuracy of the recommendation system, and the speed/performance that the system can provide recommendations to users. This project aims to explore this, by implementing different types of collaborative filtering techniques to recommend food to users based on their personal preferences and previous behaviours.


\section{What's On The Menu Overview}

What's On The Menu (WOTM) is an existing system that aims at providing personalised food recommendations to users. It is currently in development and was created by John Clegg. WOTM consists of two components: WOTM Web and WOTM API. WOTM Web is the front-end of the application that deals with interactions in the client browser. WOTM API is the back-end of the application that deals with the application logic and management of data. We will be extending the existing system by creating an additional component called WOTM Recommendations. This component will connect to the other components to provide recommendations to the users, and will manage anything related to the recommendation system. 

\section{Motivation}
\todo{Add motivation}
\section{Project Goals}
\todo{Add project goals}
\section{Address this: (From feedback)}
Not clear what the novel or hard part of the work is. Recommender systems already exist. I doubt that a recommender system for food requires a different algorithm than other subjects. Make an argument to this.
Focus is more on backend, or the overall system? If that is the case, it should be more clear in introduction. Why is this challenging? 
what parts am I going to write?
For the final report, please emphasize your particular focus and what the hardest problems are. 