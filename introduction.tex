\chapter{Introduction}\label{C:intro}

\todo{Pondy's Writing lecture}
\begin{enumerate}
 \item dont use whilst (Pondy recommended use ``when" instead, but depends on context)
 \item introduce the problem in the first sentence (find good items)
 \item get ideas down first, edit later
 \item do not edit sentences as you try and get down the ideas as they may not be used
 \item write abstract last, and intro second to last
 \item high level, then more detailed 
 \item can have a glossary
 \item have the ideas been presented in a logical order? context, problem, solution etc.
\end{enumerate}

With the vast food options around us, it is often difficult to decide what to eat. On some days we may feel like eating our favourite foods, but on other days we could feel more explorative, wanting to try something new. There are many factors that determine our decisions on what we want to eat such as our preferences, our current mood, and what food providers are nearby. Other factors that may contribute is the price of the food dish, the cuisine type, the meat type, and so forth. Additionally, a subset of users have food intolerances which restrict their food options, further making the food decision process difficult.

\todo{Talk about the Long Tail problem?}
The Long Tail problem in the context of recommender
systems has been addressed previously in [3] and [4]. In
particular, [3] analyzed the impact of recommender systems on
sales concentration and developed an analytical model of
consumer purchases that follow product recommendations
provided by a recommender system. The recommender system
follows a popularity rule, recommending the bestselling products
to all consumers, and they show that the process tends to increase
the concentration of sales. As a result, the treatment is somewhat
akin to providing product popularity information. The model in
[3] does not account for consumer preferences and their
incentives to follow recommendations or not. Also [3] studied the
effects of recommender systems on sales concentration and did
not address the problem of improving recommendations for the
items in the Long Tail, which constitutes the focus of this paper.
In [4], a related question has been studied: to which extent
recommender systems account for an increase in the Long Tail of
the sales distribution. [4] shows that recommender systems
increase firm’s profits and affect sales concentration. 



\todo{Check this again (brain dump)}
Often at times, it can be difficult to find good food dishes to eat. There are several factors that can influence our decisions. These factors must be taken into account a recommender system that is able to provide personalised recommendations to each user. Taste is subjective, people are different. The main goal of this project is to present a base recommender system that is able to ``Find Good Items" for each user, providing personalised recommendations that can be integrated for the application What's On The Menu (WOTM). The difficulty in this project, lays within the factors that determine a ``Good Item" for a user, and providing a recommender system that meets the needs of the users. The first is to find an algorithm that will be able to achieve this goal. There are a vast range of techniques that are used for recommendations, however, each algorithm fits a specific context, and relies on conditions that are needed. There are always trade-offs that need to be considered. Factors that should be considered are the scalability, the accuracy, the cold start problem, the sparsity, and the computation complexity all while considering the user experience (keeping in mind that users must be able to achieve their goal of finding good items.) So what defines a good item? Is it the novelty? the contents of the food dishes? how others feel about the dish? the price and so on? In addition to this, a main problem is not having enough data from users. How can we provide recommendations to the users when they do not indicate enough information to provide personalised recommendations? Another issue is being able to recommender system that can integrate with the existing WOTM application. In addition, there lacks research in the recommender field of systems that look at only using Binary data collected from users. Most research looks at a more fine grained rating system such as from 1-10 stars. Using only binary data for ratings is a problem in itself. Despite this, the recommender system evaluation process is a problem in itself. How do we know that the recommender system meets our needs?
 
\todo{good reasons \cite{memorybased} \citeauthor{memorybased}} 
 Every year several new techniques are proposed and yet it is not clear which of the techniques work best and under what conditions. 
The prediction accuracy of the different algorithms depends on the number of users,
the number of items, and density, where the nature and degree of dependence differs
from algorithm to algorithm.
3. There exists a complex relationship between prediction accuracy, its variance, computation
time, and memory consumption that is crucial for choosing the most appropriate
recommendation system algorithm.



\section{Address this: (From feedback)}
Not clear what the novel or hard part of the work is. Recommender systems already exist. I doubt that a recommender system for food requires a different algorithm than other subjects. Make an argument to this.
Focus is more on backend, or the overall system? If that is the case, it should be more clear in introduction. Why is this challenging? 
what parts am I going to write?
For the final report, please emphasize your particular focus and what the hardest problems are. 

\section{Motivation}

There are several problems in the recommendation domain. 
\todo{There is limited research on binary recommender systems}

\todo{Add motivation}
\subsection{Algorithm is dependent on each use case}
\subsubsection{Data that can be collected}
\subsection{Find Good Items}
\subsection{Novelty/Serendipity}
\subsection{Commercial Product that can be iterated in the future.}
\section{Project Objectives}
\todo{Add project goals}
\subsection{\todo{Sparsity}}
\subsection{\todo{Scalability}}
\subsection{\todo{Cold Start}}
\subsection{\todo{Computation Complexity}}


What’s On The Menu (WOTM) is a web-centric mobile application that is focused on recommending food dishes to users based on their personal preferences \todo{(e.g. \_, \_, \_)}, previous actions such as likes/dislikes, and other factors such as location, price, and so forth. These personalised recommendations are important as it saves time for users and allows users to explore preferred food options they otherwise might not have discovered alone. 

\todo{Takes too long to introduce the problem}
Existing applications such as Yelp and Foursquare have similar concepts but lack personalised recommendations for food dishes. These applications focus on a broader scope such as popular restaurants, coffee places, activities, and so forth. This can often lead to a cluttered interface, affecting the user experience, and may provoke confusion in users. Food spotting is an application that focuses on food dishes but does not provide personalised recommendations to users. Instead, they show popular food dishes and focus on crowd-sourcing, relying on users to upload food dishes. What's On The Menu (WOTM) differs as the goal is to recommend personalised food recommendations to the low level granularity of a food dish. Specifically, the personalised recommendations will be provided through a recommender system aimed towards the use case of finding ``Good Items" based on the user. This report, specifically focuses on a type of recommender algorithm called Collaborative filtering. 

Collaborative filtering is a popular recommendation technique used to recommend items to users based on previous rating patterns and the behaviours of others \cite{itembased, schafer2007collaborative, survey}. 

\todo{State explicit goals}
The aim of this project is to provide a base recommendation engine that will be able to provide recommendations for the web-centric mobile application ``What's On The Menu" (WOTM). In particular, the project mainly focuses on Collaborative filtering algorithms, suitable for the application to be used in a commercial environment. The main focus of the project is on the recommendation engine. This involves the algorithms that will be used, the data that will be collected, and how the existing WOTM application will communicate with the recommendation engine. 
For clarification, we explicitly state the goals of the project in the following points:
\todo{redo all of this}
\begin{itemize}
	\item{Consider a recommendation system that is able to provide accurate recommendations to users for the use case of ``Find Good Items"}
	\item{Consider the scalability of the algorithm for future iterations}
	\item{Consider the efficiency of the algorithm for future iterations}
	\item{Consider the sparsity of the algorithm for future iterations}
    \item{Compare collaborative filtering algorithms to find the a suitable recommendation algorithm}
	\item{Given binary explicit rating data, provide accurate recommendations}
\end{itemize}

% The goal of this project is to find which types of collaborative filtering algorithms best suit the WOTM application. Suitability of these algorithms will be considered on factors that are important for WOTM. The recommender system will have to account for a range of factors from the user experience, the accuracy of the recommendation system, and the speed/performance that the system can provide recommendations to users. This project aims to explore this, by implementing different types of collaborative filtering techniques to recommend food to users based on their personal preferences and previous behaviours.

\section{What's On The Menu Overview}

What's On The Menu (WOTM) is a system that aims at providing personalised food recommendations to users. It is currently in development and was created by John Clegg \todo{from Summer of Tech}. WOTM consists of two components: WOTM Web and WOTM API. WOTM Web is the front-end of the application that deals with interactions in the client browser. WOTM API is the back-end of the application that deals with the application logic and management of data. I will be extending the existing system by creating an additional component called WOTM Recommendations. This component will connect to the other components to provide recommendations to the users, and will manage anything related to the recommendation system. 

