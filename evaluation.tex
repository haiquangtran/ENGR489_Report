\chapter{Evaluation}\label{C:evaluation}

\section{Experimental Objectives}
In this paper, we aim to compare a latent factor approach to collaborative filtering with a item-based collaborative filtering approach and a hybrid collaborative filtering approach. The main objectives are to answer the following questions:
\begin{itemize}
	\item{Do dislike events add value to the results to a latent factor model CF approach}
	\item{What are the best variations of using like and dislike events for a latent factor CF approach}
	\item{Do using user preferences add value to collaborative filtering}
	\item{Is item similarity CF more accurate than user based for latent factor models}
	\item{Is hybrid CF more accurate than user based for latent factor models}
	\item{Does filtering recommendations by a black list more accurate for latent factor models}
	\item{Does filtering based on preferences of the user, more accurate for latent factor models}
\end{itemize}

\subsection{Experimental Validity}

\subsection{Measurement Platform}

\section{Experimental Design}


\section{Methods?}

\subsection{User Dataset}

% \cite{martin2009recsys}
% Examples of this problem
% include the lack of standard treatment of items for which
% the recommender is unable to make a prediction. The broader
% goal of user-centered holistic evaluation, including A/B testing
% of the short- and long-term effects of recommendation differences,
% is still met by only a few research groups and companies
% that have the live systems and resources for such evaluation.

% Deploying innovative recommenders is still too hard, and there is a substantial need for research platforms where innovations can be tested without first building up a community of thousands of users. 
\todo{Show proportions of data}

\section{Experimental Setup}

\section{Issues}

\section{Offline Evaluation}

\subsection{Data Partitioning}

\subsection{Parameter Tuning}

\subsection{Offline Metrics}

\subsubsection{Binary List}

\subsubsection{Classification Accuracy}
\todo{do this}
Classification Accuracy Metrics. Classification metrics measure the
frequency with which a recommender system makes correct or incorrect decisions
about whether an item is good. Classification metrics are thus appropriate
for tasks such as Find Good Items when users have true binary preferences.
When applied to nonsynthesized data in offline experiments, classification
accuracy metrics may be challenged by data sparsity. The problem occurs when
the collaborative filtering system being evaluated is generating a list of top
recommended items. When the quality of the list is evaluated, recommendations
may be encountered that have not been rated. How those items are treated in
the evaluation can lead to certain biases.
One approach to evaluation using sparse data sets is to ignore recommendations
for items for which there are no ratings. The recommendation list is first
processed to remove all unrated items. The recommendation task has been altered
to “predict the top recommended items that have been rated.” In tasks
where the user only observes the top few recommendations, this could lead to
inaccurate evaluations of recommendation systems with respect to the user’s task. The problem is that the quality of the items that the user would actually
see may never be measured.

In essence, we are measuring how well the system
can identify items that the user was already aware of. This evaluation approach
may result in collaborative filtering algorithms that are biased towards obvious,
nonnovel recommendations or perhaps algorithms that are over fitted—fitting
the known data perfectly, but new data poorly.

Classification accuracy metrics do not attempt to directly measure the ability
of an algorithm to accurately predict ratings. Deviations from actual ratings
are tolerated, as long as they do not lead to classification errors. The particular
metrics that we discuss are Precision and Recall and related metrics and ROC.


\subsubsection{Precision \& Recall}
\todo {precision and recall}

Recall, in its purest sense, is almost always impractical to measure in a
recommender system. In the pure sense, measuring recall requires knowing
whether each item is relevant; for a movie recommender, this would involve
asking many users to view all 5000 movies to measure how successfully we recommend
each one to each user

Perhaps a more appropriate way to approximate precision and recall would
be to predict the top N items for which we have ratings. That is, we take a
user’s ratings, split them into a training set and a test set, train the algorithm
on the training set, then predict the top N items from that user’s test set. If we
assume that the distribution of relevant items and nonrelevant items within
the user’s test set is the same as the true distribution for the user across all
items, then the precision and recall will be much closer approximations of the
true precision and recall. This approach is taken in Basu et al. [1998].
In information retrieval, precision and recall can be linked to probabilities
that directly affect the user. If an algorithm has a measured precision of 70%,
then the user can expect that, on average, 7 out of every 10 documents returned
to the user will be relevant. Users can more intuitively comprehend the meaning
of a 10\% difference in precision than they can a 0.5-point difference in mean
absolute error

One of the primary challenges to using precision and recall to compare different
algorithms is that precision and recall must be considered together to
evaluate completely the performance of an algorithm.

Precision alone at a single search length or a single recall level can be appropriate
if the user does not need a complete list of all potentially relevant
items, such as in the Find Good Items task. If the task is to find all relevant
items in an area, then recall becomes important as well. However, the search
length at which precision is measured should be appropriate for the user task
and content domain.


\subsubsection{ROC Curves}

\subsubsection{Thresholding}

\section{Online Evaluation}

\section{Results}

\section{Limitations}

\section{Discussion}


