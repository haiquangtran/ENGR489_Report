%% $RCSfile: proj_report_outline.tex,v $
%% $Revision: 1.2 $
%% $Date: 2010/04/23 02:40:16 $
%% $Author: kevin $

\documentclass[11pt
, a4paper
, twoside
, openright
]{report}

\usepackage{float} % lets you have non-floating floats

\usepackage{url} % for typesetting urls
\usepackage[numbers]{natbib}
\usepackage[toc,page]{appendix}
\usepackage{pdfpages}
\usepackage{color}
\usepackage{amsmath}
\usepackage{graphicx}
\usepackage{xcolor}
\usepackage{multirow}
\newcommand\todo[1]{\textcolor{red}{#1}}

\usepackage{listings}
\usepackage{color}

\definecolor{dkgreen}{rgb}{0,0.6,0}
\definecolor{gray}{rgb}{0.5,0.5,0.5}
\definecolor{mauve}{rgb}{0.58,0,0.82}

\lstset{frame=tb,
  language=Ruby,
  aboveskip=3mm,
  belowskip=3mm,
  showstringspaces=false,
  columns=flexible,
  basicstyle={\small\ttfamily},
  numbers=none,
  numberstyle=\tiny\color{gray},
  keywordstyle=\color{blue},
  commentstyle=\color{dkgreen},
  stringstyle=\color{mauve},
  breaklines=true,
  breakatwhitespace=true,
  tabsize=3
}

%
%  We don't want figures to float so we define
%
\newfloat{fig}{thp}{lof}[chapter]
\floatname{fig}{Figure}

%% These are standard LaTeX definitions for the document
%%                            
\title{Food Recommendations}
\author{Hai Quang Tran : 300224467}

%% This file can be used for creating a wide range of reports
%%  across various Schools
%%
%% Set up some things, mostly for the front page, for your specific document
%
% Current options are:
% [ecs|msor]              Which school you are in.
%
% [bschonscomp|mcompsci]  Which degree you are doing
%                          You can also specify any other degree by name
%                          (see below)
% [font|image]            Use a font or an image for the VUW logo
%                          The font option will only work on ECS systems
%
\usepackage[ecs,image]{vuwproject}

% You should specifiy your supervisor here with
\supervisors{John Clegg, Dr Xiaoying Gao, Dr Ian Welch}
% use \supervisors if there is more than one supervisor

% Unless you've used the bschonscomp or mcompsci
%  options above use
\otherdegree{Bachelor of Engineering with Honours}
% here to specify degree

% Comment this out if you want the date printed.
\date{}

\begin{document}

% Make the page numbering roman, until after the contents, etc.
\frontmatter

%%%%%%%%%%%%%%%%%%%%%%%%%%%%%%%%%%%%%%%%%%%%%%%%%%%%%%%

%%%%%%%%%%%%%%%%%%%%%%%%%%%%%%%%%%%%%%%%%%%%%%%%%%%%%%%

\begin{abstract}

This project explores recommendation techniques to integrate a recommender system into the web-centric mobile application What's On The Menu (WOTM). This recommendation system should be able to provide personalised recommendations to users based on the use case of ``Find Good Items" seen as items that a user will prefer based on their personal tastes towards food.

A user study has been conducted collecting a dataset containing 91 users, 100 food items, 4207 positive ratings, and 1638 negative ratings for an offline evaluation for recommender systems. A baseline predictor system recommending non-personalised items based on popularity was compared to four recommender systems, providing personalised recommendations. These systems were an Item-based SVD approach, a Single SVD approach, a Dual SVD approach, and a Hybrid CF approach. 

Overall, the Hybrid CF system was able to outperform every other system in terms of Accuracy, Precision, Area Under the Curve (ROC). In particular, the Hybrid CF approach was able to provide personalised recommendations at an accuracy of 87.06\%, outperforming the popularity baseline which provided non-personalised recommendations at an accuracy of 73.84\%, fulfilling the goal of ``Find Good Items".  

% The system will be tested using evaluation techniques (yet to be identified) on real user data. The user data will be collected during the course of Trimester 2, and will test the accuracy in addition to the performance and scalability of the system. This is yet to be determined. 
% The project looks at how a recommender system can be used to recommend food dishes to users based on personal food preferences, and previous binary rating patterns (Like/Dislike). 
\end{abstract}

%%%%%%%%%%%%%%%%%%%%%%%%%%%%%%%%%%%%%%%%%%%%%%%%%%%%%%%

\maketitle

% \include{acknowledge}

\tableofcontents

% we want a list of the figures we defined
\listoffigures

%%%%%%%%%%%%%%%%%%%%%%%%%%%%%%%%%%%%%%%%%%%%%%%%%%%%%%%

\mainmatter

%%%%%%%%%%%%%%%%%%%%%%%%%%%%%%%%%%%%%%%%%%%%%%%%%%%%%%%

% individual chapters included here
\chapter{Introduction}\label{C:intro}

\todo{Pondy's Writing lecture}
\begin{enumerate}
 \item dont use whilst (Pondy recommended use ``when" instead, but depends on context)
 \item introduce the problem in the first sentence (find good items)
 \item get ideas down first, edit later
 \item do not edit sentences as you try and get down the ideas as they may not be used
 \item write abstract last, and intro second to last
 \item high level, then more detailed 
 \item can have a glossary
 \item have the ideas been presented in a logical order? context, problem, solution etc.
\end{enumerate}

With the vast food options around us, it is often difficult to decide what to eat. On some days we may feel like eating our favourite foods, but on other days we could feel more explorative, wanting to try something new. There are many factors that determine our decisions on what we want to eat such as our preferences, our current mood, and what food providers are nearby. Other factors that may contribute is the price of the food dish, the cuisine type, the meat type, and so forth. Additionally, a subset of users have food intolerances which restrict their food options, further making the food decision process difficult.

\todo{Talk about the Long Tail problem?}
% The Long Tail problem in the context of recommender
% systems has been addressed previously in [3] and [4]. In
% particular, [3] analyzed the impact of recommender systems on
% sales concentration and developed an analytical model of
% consumer purchases that follow product recommendations
% provided by a recommender system. The recommender system
% follows a popularity rule, recommending the bestselling products
% to all consumers, and they show that the process tends to increase
% the concentration of sales. As a result, the treatment is somewhat
% akin to providing product popularity information. The model in
% [3] does not account for consumer preferences and their
% incentives to follow recommendations or not. Also [3] studied the
% effects of recommender systems on sales concentration and did
% not address the problem of improving recommendations for the
% items in the Long Tail, which constitutes the focus of this paper.
% In [4], a related question has been studied: to which extent
% recommender systems account for an increase in the Long Tail of
% the sales distribution. [4] shows that recommender systems
% increase firm’s profits and affect sales concentration. 



\todo{Check this again (Massive massive brain dump)}
Often at times, it can be difficult to find good food dishes to eat. There are several factors that can influence our decisions. These factors must be taken into account a recommender system that is able to provide personalised recommendations to each user. Taste is subjective, people are different. The main goal of this project is to present a base recommender system that is able to ``Find Good Items" for each user, providing personalised recommendations that can be integrated for the application What's On The Menu (WOTM). The difficulty in this project, lays within the factors that determine a ``Good Item" for a user, and providing a recommender system that meets the needs of the users. The first is to find an algorithm that will be able to achieve this goal. There are a vast range of techniques that are used for recommendations, however, each algorithm fits a specific context, and relies on conditions that are needed. There are always trade-offs that need to be considered. Factors that should be considered are the scalability, the accuracy, the cold start problem, the sparsity, and the computation complexity all while considering the user experience (keeping in mind that users must be able to achieve their goal of finding good items.) So what defines a good item? Is it the novelty? the contents of the food dishes? how others feel about the dish? the price and so on? In addition to this, a main problem is not having enough data from users. How can we provide recommendations to the users when they do not indicate enough information to provide personalised recommendations? Another issue is being able to recommender system that can integrate with the existing WOTM application. In addition, there lacks research in the recommender field of systems that look at only using Binary data collected from users. Most research looks at a more fine grained rating system such as from 1-10 stars. Using only binary data for ratings is a problem in itself. Despite this, the recommender system evaluation process is a problem in itself. How do we know that the recommender system meets our needs?
 
\todo{good reasons \cite{memorybased} \citeauthor{memorybased}} 
 Every year several new techniques are proposed and yet it is not clear which of the techniques work best and under what conditions. 
The prediction accuracy of the different algorithms depends on the number of users,
the number of items, and density, where the nature and degree of dependence differs
from algorithm to algorithm.
3. There exists a complex relationship between prediction accuracy, its variance, computation
time, and memory consumption that is crucial for choosing the most appropriate
recommendation system algorithm.



\section{Address this: (From feedback)}
Not clear what the novel or hard part of the work is. Recommender systems already exist. I doubt that a recommender system for food requires a different algorithm than other subjects. Make an argument to this.
Focus is more on backend, or the overall system? If that is the case, it should be more clear in introduction. Why is this challenging? 
what parts am I going to write?
For the final report, please emphasize your particular focus and what the hardest problems are. 

\section{Motivation}

There are several problems in the recommendation domain. 
\todo{There is limited research on binary recommender systems}

\todo{Add motivation}
\subsection{Algorithm is dependent on each use case}
\subsubsection{Data that can be collected}
\subsection{Find Good Items}
\subsection{Novelty/Serendipity}
\subsection{Commercial Product that can be iterated in the future.}
\section{Project Objectives}
\todo{Add project goals}
\subsection{\todo{Sparsity}}
\subsection{\todo{Scalability}}
\subsection{\todo{Cold Start}}
\subsection{\todo{Computation Complexity}}


What’s On The Menu (WOTM) is a web-centric mobile application that is focused on recommending food dishes to users based on their personal preferences \todo{(e.g. \_, \_, \_)}, previous actions such as likes/dislikes, and other factors such as location, price, and so forth. These personalised recommendations are important as it saves time for users and allows users to explore preferred food options they otherwise might not have discovered alone. 

\todo{Takes too long to introduce the problem}
Existing applications such as Yelp and Foursquare have similar concepts but lack personalised recommendations for food dishes. These applications focus on a broader scope such as popular restaurants, coffee places, activities, and so forth. This can often lead to a cluttered interface, affecting the user experience, and may provoke confusion in users. Food spotting is an application that focuses on food dishes but does not provide personalised recommendations to users. Instead, they show popular food dishes and focus on crowd-sourcing, relying on users to upload food dishes. What's On The Menu (WOTM) differs as the goal is to recommend personalised food recommendations to the low level granularity of a food dish. Specifically, the personalised recommendations will be provided through a recommender system aimed towards the use case of finding ``Good Items" based on the user. This report, specifically focuses on a type of recommender algorithm called Collaborative filtering. 

Collaborative filtering is a popular recommendation technique used to recommend items to users based on previous rating patterns and the behaviours of others \cite{itembased, schafer2007collaborative, survey}. 

\todo{State explicit goals}
The aim of this project is to provide a base recommendation engine that will be able to provide recommendations for the web-centric mobile application ``What's On The Menu" (WOTM). In particular, the project mainly focuses on Collaborative filtering algorithms, suitable for the application to be used in a commercial environment. The main focus of the project is on the recommendation engine. This involves the algorithms that will be used, the data that will be collected, and how the existing WOTM application will communicate with the recommendation engine. 
For clarification, we explicitly state the goals of the project in the following points:
\todo{redo all of this}
\begin{itemize}
	\item{Consider a recommendation system that is able to provide accurate recommendations to users for the use case of ``Find Good Items"}
	\item{Consider the scalability of the algorithm for future iterations}
	\item{Consider the efficiency of the algorithm for future iterations}
	\item{Consider the sparsity of the algorithm for future iterations}
    \item{Compare collaborative filtering algorithms to find the a suitable recommendation algorithm}
	\item{Given binary explicit rating data, provide accurate recommendations}
\end{itemize}

% The goal of this project is to find which types of collaborative filtering algorithms best suit the WOTM application. Suitability of these algorithms will be considered on factors that are important for WOTM. The recommender system will have to account for a range of factors from the user experience, the accuracy of the recommendation system, and the speed/performance that the system can provide recommendations to users. This project aims to explore this, by implementing different types of collaborative filtering techniques to recommend food to users based on their personal preferences and previous behaviours.

\section{What's On The Menu Overview}

What's On The Menu (WOTM) is a system that aims at providing personalised food recommendations to users. It is currently in development and was created by John Clegg \todo{from Summer of Tech}. WOTM consists of two components: WOTM Web and WOTM API. WOTM Web is the front-end of the application that deals with interactions in the client browser. WOTM API is the back-end of the application that deals with the application logic and management of data. I will be extending the existing system by creating an additional component called WOTM Recommendations. This component will connect to the other components to provide recommendations to the users, and will manage anything related to the recommendation system. 


\chapter{Background Survey \& Related Work}\label{C:background}

\section{Collaborative Filtering}

Historically, people have relied on recommendations and mentions
from their peers or the advice of experts to support decisions and discover
new material 

collaborative filtering, a class of methods that
recommend items to users based on the preferences other users have
expressed for those items.
There is also a growing interest in problems surrounding
recommendation. Algorithms for understanding and predicting user
preferences do not exist in a vacuum — they are merely one piece of
a broader user experience. A recommender system must interact with
the user, both to learn the user’s preferences and provide recommendations;
these concerns pose challenges for user interface and interaction
design. Systems must have accurate data from which to compute their
recommendations and preferences, leading to work on how to collect
reliable data and reduce the noise in user preference data sets. Users
also have many different goals and needs when they approach systems,
from basic needs for information to more complex desires for privacy
with regards to their preferences.

In his keynote address at the 2009 ACM Conference on Recommender Systems,Martin \cite{martin2009recsys} argued that algorithms themselves are only a small part of the problem of providing recommendations to users. We haev a number of algorithms that work fairly well, and while there is room to refine them, there is much work to be done on user experience, data colelction, and other problems which make up the whole of the recommender experience. 

Collaborative filtering (CF) is a popular recommendation algorithm
that bases its predictions and recommendations on the ratings or behavior
of other users in the system.

There are other methods for performing recommendation, such as
finding items similar to the items liked by a user using textual similarity
in metadata (content-based filtering or CBF). The focus of this survey
is on collaborative filtering methods, although content-based filtering
will enter our discussion at times when it is relevant to overcoming a particular recommender system difficulty.

\cite{schafer2007collaborative}

TODO: reword.
Collaborative filtering is a recommendation algorithm that predicts items (filtering) about the interests of a user by collecting preferences or taste information from other users (collaborating). The underlying assumption of the collaborative filtering approach is that similar users will have similar tastes, thus we can make recommendations based on this idea. 

Collaborative Filtering is a technique used in many recommender systems. The main idea behind collaborative filtering is to analyse previous user behaviour such as their previous ratings on items to find trends in these patterns. By learning these trends this technique is able to identify new user item relationships to recommend items that the user may have not seen yet. 


However, this means that the algorithm is reliant on previous user behaviour to give good recommendations. This makes collaborative filtering difficult to address new products and users. This problem is referred to as the 'Cold Start' problem where there are not many users who have rated any items yet. (Content based filtering is superior in this context). However, the advantage of Collaborative filtering is that does not require domain knowledge of the products and users. It purely focuses on the previous user rating actions, and recommendations are generally more accurate than Content based filtering \cite{koren2009matrix}. 

There are two main areas that encompass collaborative filtering, these areas are neighborhood methods and latent factor models which will be explained in detail further on. 

% \section{Challenges}
% \subsection{Cold Start}
% \subsection{Data Sparsity}
% \subsection{Synonyms}
% \subsection{Scalability}
% \subsection{Grey Sheep}
% \subsection{Shilling attacks}
% \subsection{Diversity and the Long Tail}

\section{Neighbourhood Methods}


Neighbourhood methods cluster together items or users that are most similar to each other according to similarity measures. Users and items are treated 

\subsection{User Based Collaborative Filtering}

It was first introduced in the GroupLens Usenet article recommender \cite{grouplens}. 

User based collaborative filtering uses a similarity measure to find users that are similar to each other, and then recommends food based on this metric. User based collaborative filtering methods aim at finding users that are similar to one another, and recommending items to them based on each of their previous behaviours. There is high chance that similar users will prefer similar items. For example, two users may like eating spicy food. the first user has indicated that they like eating chicken vindaloo and hot chilli peppers. The second user has indicated that they like eating hot chilli peppers. Therefore, since these two users previous behaviours are similar as they both have indicated their enjoyment of hot chilli peppers, they are to be considered similar. We can recommend to the second user that they should try chicken vindaloo, since the first user liked it and the two users have similar tastes.  


\subsection{Item Based Collaborative Filtering}

Another approach that is commonly used is called item based collaborative filtering. Instead of finding users that are similar to one another, item based collaborative filtering focuses on previous behaviour of the user, and recommends similar new items based on what items the user previously liked. Item based collaborative filtering differs to user based collaborative filtering because it uses a similarity measure to find similar items to items that the user has already indicated interest in. For example, a user could have previously liked the following dishes: a chicken sandwich, chicken nuggets, and chicken soup. Item based collaborative filtering will find similar items to this user by using a similarity measure on their previous items they indicated they liked. By using this technique, it may recommend new items to the user such as chicken salad, or a chicken burger since all those items contain chicken. This is the main idea behind item based collaborative filtering. 

\cite{schafer2007collaborative}

 Rather than using
similarities between users’ rating behavior to predict preferences, item–
item CF uses similarities between the rating patterns of items. If two
items tend to have the same users like and dislike them, then they are
similar and users are expected to have similar preferences for similar
items. In its overall structure, therefore, this method is similar to earlier
content-based approaches to recommendation and personalization, but
item similarity is deduced from user preference patterns rather than
extracted from item data.


\subsubsection{Slope One}

\subsection{Challenges}
A key problem of collaborative filtering is how to combine and weight the preferences of user neighbors. Sometimes, users can immediately rate the recommended items. As a result, the system gains an increasingly accurate representation of user preferences over time.

\subsection{Similarity Measures}
\subsubsection{Pearson's Coefficient}
\subsubsection{Euclidean Distance \& Manhattan Distance}
\subsubsection{Cosine Similarity}
\subsubsection{Jaccard Index}

\section{Latent Factor Methods}

Latent factor models look at analysing the previous actions of users and creating a feature vectors for items and users. These item and user features are used to predict future items that the user may like. 

\subsection{Singular Value Decomposition}
\subsubsection{Stochastic Gradient Descent}
\subsubsection{Alternating Least Squares}


\section{Literature Review}

The term 'collaborative filtering' was first introduced in \citeyear{goldberg1992using} by \citeauthor{goldberg1992using} to describe the technique used in Tapestry, one of the earliest known recommender systems \cite{koren2009matrix,  goldberg1992using, itembased, survey}.

Tapestry \cite{goldberg1992using} was created to handle electronic documents and used manual collaborative filtering, allowing users to query information based on others opinions about the documents. These opinions were in the form of annotations or replies which users were encouraged to make on documents to increase probability of relevant results returned from queries \cite{schafer2007collaborative}. Tapestry relied on opinions from a small community such as an office workgroup, where each person's opinion was trusted. Larger communities could not rely on every person knowing each other, leading to new collaborative filtering techniques being developed \cite{itembased}. 

More recommender systems emerged as value was seen in the potential to increase sales from recommendations - customers may purchase suggested items that they might not have seen otherwise \cite{schafer2007collaborative}. Perhaps the most popular recommender system in the late 1990's was used in Amazon.com, collecting user purchase history, browsing history, and recently viewed items to recommend items that the user may buy \cite{schafer2007collaborative}. Other recommender systems consisted of Jester \cite{goldberg} for jokes, and Ringo \cite{ringo} for music.
% /cite{schafer2007collaborative, toward}
GroupLens \cite{grouplens} were the first to introduce a neighbourhood collaborative filtering technique. Building upon the Tapestry concept, GroupLens created an automated user based collaborative filtering technique for recommending Usenet articles that users may be interested in. The advantage of neighbourhood methods is that they are intuitive, easy to implement, and produce highly effective results \cite{survey, scalable}. Despite providing accurate recommendations, user-based collaborative filtering techniques were computed in real time and performance would degrade as more users and items were added to the system, leading to scalability and performance issues \cite{dimension, itembased, evaluationitem}.

This required collaborative filtering techniques that could easily scale and still produce high quality recommendations leading to the exploration of item-based collaborative filtering. Item-based collaborative filtering techniques were developed to address scalability limitations of the user-based techniques \cite{survey}. \citeauthor{itembased} analyzed various item-based recommendation algorithms, computing item-item similarities and comparing the accuracy with traditional KNN user based collaborative filtering techniques \cite{itembased}. \citeauthor{itembased} found that items remained fairly static in the system, whereas user behaviours and preferences would often change. Because items were found mostly static, it meant precomputation could occur for item similarities. By having precomputed item similarities, traditional item-based collaborative filtering can then be applied, thus performance and scalability would be increased \cite{scalable}.

Other techniques such as model based collaborative filtering have been investigated to overcome the performance and scalability issues. Well known model based techniques include Bayesian belief nets \cite{baysian}, clustering models \cite{clustering}, and latent semantic models \cite{latent}. These models are based on learning patterns from users previous actions to predict new items, and are expensive but can be built offline allowing high scalability. The resulting model is "very small, very fast, and essentially as accurate as nearest neighbor methods" \cite{itembased}. \citeauthor{itembased} found Bayesian networks to be practical in the context where user "preferences change slowly with respect to the model" \cite{itembased}. However, these models are not suitable for environments where the user preference model should be updated rapidly or frequently. Since model based approaches do not have to compute similarity measures to form neighbourhoods, they tend to produce faster recommendations and outperform neighbourhood models in terms of accuracy of recommendations \cite{toward, itembased}. 

Although collaborative filtering is considered to be one of the most successful approaches to recommender systems \cite{survey, toward}, they suffer from the problem of data sparsity \cite{toward, survey, itembased, koren2009matrix, koren2011, dimension}. Data sparsity is when only a small subset of user ratings on items are recorded, leading to a insufficient number of ratings to produce accurate recommendations. Data sparsity specifically tends to appear in the 'cold start' problem, where new items or new users are entered into the system, but not enough information is supplied to produce accurate recommendations since recommendations are based on common items or users \cite{survey}.

To alleviate this problem, hybrid approaches were investigated that combined collaborative filtering and other recommender techniques such as content based filtering. \citeauthor{toward} suggested creating user profiles such that demographic information could be included in similarity measures to provide extra content to find similar users or items. This effectively makes use of content-based filtering where recommendations are produced based on the content and attributes of the items, learning what attributes the user likes \cite{toward}. Well-known hybrid techniques include content-boosted collaborative filtering \cite{hybrid}, and personality diagnosis \cite{hybrid2, survey}. Hybrid approaches were implemented to address the limitations of collaborative filtering and content-based filtering techniques \cite{toward}, but have increased complexity leading to more expensive computations. Additionally, external information is needed about the content of the items which may not be available, thus making hybrid approaches impractical in certain scenarios \cite{survey}. \citeauthor{dimension} found a different approach that used dimension reduction techniques such as Singular Value Decomposition, making sparse rating models more dense by reducing the dimensionality of the product space, thereby condensing the modelled ratings of users and producing less missing information \cite{dimension}. 

In 2006, the Netflix Prize competition attracted interest in the field of recommender systems \cite{survey}. Netflix offered a \$ 1 million dollar prize to the first team to improve their movie recommender system by 10\%. This attracted interest in the research field of recommender systems. The team "BellKor in Pragmatic Chaos" won the competition in 2009 basing their solution on a combination of latent factor models and neighbourhood models \cite{winning, survey}. These models took into account many biases which improved the predication accuracy. \citeauthor{koren2009matrix} were part of the winning team, and wrote a paper explaining how temporal effects, and user biases could be accounted for in latent factor models such as Singular Value Decomposition, making it superior to neighbourhood methods \cite{koren2009matrix}. \citeauthor{koren2011} later published a paper about their findings and solutions to the Netflix Prize competition in \cite{koren2011} and \cite{winners}.


\section{Discussion of Literature Review}

TODO: DELETE?

Although actions from users are required to provide personalised results, the advantage of collaborative filtering is that no prior knowledge or context is required about the items to provide recommendations to users. Items could be recommended to users purely based on past behaviour from others \cite{koren2009matrix, schafer2007collaborative}. 

For this reason, collaborative filtering is preferred over content based filtering, a technique where items are recommended based on learning users preferences from previous items that users liked, and recommending similar items based on the attributes of those items. Additionally, content based filtering is prone to recommending only a small subset of items, since it recommended items based on the attributes that the user liked. Traditional collaborative filtering is able to address this concern since attributes are ignored, and recommendations are based on actions of other users, providing a range of recommendations that are not restricted to specific attributes, generally being more accurate \cite{koren2009matrix}.  In content based filtering it is also hard to extract features or would have to be manually put in which may not be practical \cite{toward}. Another problem is overspecialization, recommending only similar items to users thus restricting variety \cite{toward}.



It is evident that there has been an abundance of existing research on collaborative filtering techniques and ways to improve the prediction accuracy. However, \citeauthor{schafer2007collaborative} states these factors alone, do not contribute to making a good recommender system \cite{schafer2007collaborative}. Instead, \citeauthor{schafer2007collaborative} states that recommendation is not a "one-size-fits-all problem"  \cite{schafer2007collaborative}. Specific tasks, information needs, and item domains represent unique problems for recommenders, and design and evaluation of recommenders needs to be done based on the user tasks to be supported" \cite{schafer2007collaborative}. Similarly, \citeauthor{martin2009recsys} argues that the recommender algorithms is only one factor from many for providing recommendations to users. \citeauthor{martin2009recsys} explains that the user experience, data collection, and other problems which make up the whole of the recommender experience need to be considered \cite{schafer2007collaborative, martin2009recsys}. \citeauthor{interface} concluded in \cite{interface} that much of the accuracy problem has been solved in recommender systems, however delivering these accurate predictions to users in a way that creates the "best experience for them remains an open problem" \cite{interface}. 

For this reason, this project focuses on the goal of providing a recommender system that fits the needs for the "What's On The Menu" application. This involves considering how users ratings will be collected, the user experience, the recommendation process, and what factors are considered to be important in the recommendation process such as scalability, prediction accuracy, and performance.








TODO: remove this? Do I talk about this stuff here?


Existing research also focuses on the scalability and performance of these collaborative filtering techniques. In terms of scalability, the "What's On The Menu" application is not expected to contain anywhere near the number of users or items as existing recommender systems used by Netflix, Facebook, and Google. For this reason, neighbourhood methods may be a good choice, as they provide for the ability to explain the reasoning behind the recommendations, as well as give good accurate results. Scalability should not be a top concern with this application. Therefore, real time computation may be a factor to consider. 

The main focus of the recommender system will be on how to make the user experience as easy as possible for the users. This will enable the collection of user data to provide these recommendations. Another focus will be on the performance in which the recommender system can provide recommendations to the users. In this case, prediction accuracy may be less important than the speed and the performance of recommendations being produced. If the recommendation process is slow, then users will be less likely to continue using the application. On the other hand, if prediction accuracy is not very good, then users will be recommended items that they may potentially not like. A fair trade-off must be considered. 

% + main challenges, sparsity \cite{survey, dimension}, scalability, synonymy \cite{dimension}, gray sheep, shilling attacks, privacy protection, etc \cite{survey}.

\section{Representational State Transfer (REST)}

REST stands for REpresentational State Transfer. REST is an architectural style for building a software application that can be used to communicate to other systems through the HTTP protocol. It uses resources to represent important data that can be retrieved by other systems via methods from the HTTP Protocol (GET, POST, PUT, DELETE etc). It utilizes a client-server, and is stateless in the sense that all the data that is needed is sent through the HTTP protocol to the other system.

\section{Application Program Interface (API)}

API stands for Application Program Interface. An API describes the methods and ways in which others can interact and use these tools to build software applications. It specifies how software components should interact and how the software methods, or services behave. 
\chapter{Work Done}\label{C:work_done}

\section{Assumptions}

\section{Design Decisions}

\subsection{WOTM Recommendations Component}
TODO: Should this be in DESIGN DECISIONS???

WOTM Recommendations is an extension to the system that is a REST API handling all the data that is needed to make recommendations to users. It connects to the main wotm\_dev database and stores previous user events such as the users likes/dislikes and their preferences. It uses this data to recommend food to the users and sends it back to WOTM Web. 

The reason why we did not want to merge these events in the WOTM API is because it will then be tightly coupled to that system. By extracting it out, it means that WOTM API and WOTM Recommendations are not dependent on each other, and have their own sole purposes. WOTM API manages user and dish data, whilst WOTM recommendations manages everything to do with the recommendations. In that way, if we decide to remove recommendations, then we do not have to change code from the WOTM API, since it is loosely coupled to it. 

\subsection{Open source projects}
Spotify/Annoy, JRubyMahout, Apache Hadoop, 
\subsubsection{Apache Mahout}
\subsubsection{Recommendable}
\subsubsection{PredictionIO}
\subsubsection{GraphLab}

\subsection{Recommender System}
\subsubsection{Online Learning vs Offline Learning}

\subsection{User Model}

\subsubsection{Normalized vs Denormalized}

PostgresSQL, NoSQL, Graph Database? Neo4J

\subsubsection{Implicit vs Explicit Ratings}

\subsubsection{Boolean Ratings vs Likert Scales Ratings}

Although prediction accuracy is important, it is not the only factor that a recommender should focus on and acts as one facet in wide range of facets \cite{martin2009recsys}. \citeauthor{martin2009recsys} explains that the goal of a recommender system is to improve user experience however, designing a recommender system to fit application remains a challenge.  that user experience recommender system ratings should be highly based on the user experience \cite{martin2009recsys}. By focusing on the user experience, users will be able to easily rate food dishes they like, in turn, leading to the system collecting more recommendations from the ease of use. For this reason, the simplest model that allows users to easily rate a dish would be a simple boolean rating such as a like/dislike. The advantage is the ease of use for the simplicity, however it means that recommendations will not be as accurate as explicit rating values such as from 1-5 stars or 1-10 stars. By easing the user experience for rating food dishes it can increase data collection at the expense of accuracy. Using scaled ratings mean that we learn more about the user preferences because of the scaled factor indicating how much the user likes a dish or not. This means that model collaborative filtering techniques are able to learn what the user likes and dislikes quicker as well as leading to more accurate recommendations. 

A trade-off to consider is whether or not accurate recommendations are more important than the user experience. Since What's On The Menu aims at being a mobile application, the limitations are the small form factor that mobile phone screens have. With a scaled rating system, the user has to be shown these possible options in order to rate the dish, which may take up additional space that is not needed on such small screens. Because of this, having a 1-5 star rating system may degrade the user experience as opposed to a simple like/dislike rating system. This may lead to less data being collected and in the long term affect the recommendations that are provided to the users. But on the other hand, since it is scaled ratings, it means we are able to convey more information from each rating. This means that the recommender system is more likely to predict accurate recommendations with less ratings. 

However, one could argue that the accuracy of the ratings may not matter a great deal. For instance, a recommender system could predict two dishes the user  may like based on previous rating patterns. The first dish is predicted with a 90\% prediction accuracy, and the second dish is predicted with a 70\% prediction accuracy that user will like these dishes. But is the difference in prediction accuracy important if the user likes both dishes? As long as the recommender system has provided the user with dishes they like, the interval of accuracy between those predicted dishes do not drastically matter. In addition to this, dishes with higher predicted accuracy may seem like obvious choices of dishes that users may have already tried, whereas dishes with a lower predicted accuracy may be less obvious dishes that they may like, but have not tried yet. It should not matter how much the user likes the dish as long as the user likes the dishes that are recommended. A recommender system using boolean values would eventually reach that of using likert scale ratings.
TODO: 

Although this will happen in the long term, new values and ratings will be less accurate that join the system. 


Foursquare is an application that asks users to rate items according to a series of questions. These questions consist of "What do you like about this place?", "What is this place known for?" and so on. From these questions they are able infer a particular rating for the item, as well as collect data from the users to give more accurate recommendations. This rating system may risk users not rating the items because of the long list of questions it asks. Users may also have short attention spans and do not want to do such tasks.

For these reasons, we found that simple like/dislike events would best suit the collection of data for the recommender system because of the simplified structure which increases the user experience. The recommender system can be extended to take in additional events that may portray additional information such as a "want" indicating that a user "wants" a dish but has not yet tried it before. However, caution must be taken as more events will affect the user experience of the application. 

\section{Implementation}

\subsection{Neighbourhood Collaborative Filtering}
\subsubsection{Modified Jaccard Index}


\subsection{Latent Factor Model}
\subsubsection{Alternating Least Squares}

\subsection{Recommendation Engines}


\subsubsection{Feedback from Recommendations}

\cite{martin2009recsys}
Adapt nature of recommendations as user gains more experience with the recommender - a new user may need more verifiable recommendations and may lack the trust needed for high-risk recommendations. 
How to balance serving the individual now vs. serving the individual and community long-term. 

Part of the research challenge is to design interfaces that give users control over the recommendation process without overwhelming the user or rendering the tool too complicated for novice users. 





\chapter{Evaluation}\label{C:evaluation}

\section{Objectives}
\begin{enumerate} \label{experiment_objectives}
	\item{Can collaborative filtering provide personalised recommendations to users for the use case of ``Find Good Items"?}
	\item{Are personalised recommendations superior in accuracy than non-personalised recommendations based on the baseline predictor (item popularity)?}
	\item{Does a hybrid CF approach provide more accurate recommendations to a SVD CF approach?}
	\item{Does an item-based SVD CF approach provide more accurate recommendations than a Hybrid CF approach?}
	\item{Do correlated recommendations using binary ratings (Like/Dislike) provide more accurate recommendations than non-correlated binary ratings?}
\end{enumerate}

The objectives of this report focus on providing a base recommendation system that is able to fulfill the overall goal of ``Find Good Items" specified in Section \ref{C:intro}. To demonstrate whether the goal has been achieved, item popularity is used as a baseline predictor, comparing CF approaches to the baseline predictor which provides non-personalised recommendations. Additional objectives focus on comparing variations of latent factor CF approaches to a hybrid CF approach presented in Section \ref{algorithms}. The main objectives of the experiment are to answer the questions in Section \ref{experiment_objectives} in regard to the goal of ``Find Good Items".

\section{User Study}

The goal of the experiment is to provide a base recommendation system that provides personalised food dish recommendations to users in regard to the ``Find Good Items" use case. Collaborative filtering (CF) heavily relies on user data to provide recommendations. To capture data, a survey was given to participants asking them to rate their food preferences, indicate their food intolerances, and rate food dishes using a Likert Scale Rating system that best reflected their opinion about food dishes. This data was used in an offline experiment to evaluate latent factor CF approaches and a hybrid CF approach. Binary recommendation accuracy metrics were used to evaluate the performance of the CF algorithms in regards to the goal of ``Find Good Items", consisting of the following: recommendation accuracy, precision, recall, area under the curve (the receiving operator curve), and precision at top N (10). 

Due to time constraints and requirement of real user data for CF evaluation, an online or user study was unable to be conducted. Therefore, an offline evaluation is done to determine a suitable CF model for further online and user evaluation. 

\subsection{Participants}

A total of 91 participants were involved in the experiment. The majority of participants were students studying at the school of Engineering and Computer Science (ECS) from Victoria University of Wellington (VUW). An estimated number of 55 students were from a first year COMP102 (ECS) lecture and tutorial. An estimated number of 10 participants were in their honours or masters year in ECS recruited from the Honours CO232 lab at VUW, and an estimated number of 20 participants were from a Summer of Tech event that was held at Kelburn Campus at VUW. The remaining participants were students from around the 200 level Cotton (ECS) labs from VUW. The data collected from participants were approved by the Human Ethics Committee (HEC), and can be seen in Appendix \ref{appendix:hec}.

The ages and gender of participants were not captured due to continual changes submitted in the HEC process. It is estimated the majority of participants are students in the age range of 17-24 since the majority were recruited from the School of Engineering and Computer Science at Victoria University. It is possible a small number of participants are under or over the age range of 17-24, and may not be students. In addition, it is likely there are more male participants than female participants due to the location of recruitment.

Participants were asked to fill out surveys capturing their food preferences, food intolerances, and their opinions about food dishes. Strict food preferences were presented to participants as seen in Figure \ref{fig:strict_prefs}. The data around the participants strict food preferences collected from the survey can be seen in Table \ref{table:food_participants}:

\begin{figure}
\centering
\includegraphics[scale=0.8]{images/strict_prefs.png}
\caption{Participants in the survey were asked to rate their food preferences and their opinions about food dishes based on using a Likert Scale rating system that best reflected how they felt about the item shown. The figure above illustrates the strict food preferences that are shown to participants.}
\label{fig:strict_prefs}
\end{figure}

\begin{table}[h!]
\centering
\begin{tabular}{|l|l|} 
 \hline
 \multicolumn{2}{|c|}{Percentage of Participants with Strict Food Preferences} \\
    \hline
    \hline
    Participants & Percentage \\
     \hline\hline
     Vegetarian & 7\%\\ [0.5ex] 
     \hline
     Vegan & 5\% \\
     \hline
     Gluten Intolerant & 8\% \\
    \hline
     Diary Intolerant & 5\% \\
     \hline
     Nuts Intolerant & 0\% \\
     \hline
     Egg Intolerant & 0\% \\
     \hline
     Other Intolerance & 0\% \\
     \hline
\end{tabular}
\caption{Percentage of participants in the whole dataset that have strict food preferences.}
\label{table:food_participants}
\end{table}

\subsection{Survey}

Quantitative data about user food ratings were collected from participants in a survey. Participants were asked to rate their food preferences, their food intolerances, and their opinion about food dishes from the following Likert Scale rating system: Hate, Dislike, Neutral, Like, and Love. Food preferences refer to preferences such as meat, pastry, soup, noodles etc. Food intolerances refer to intolerance to gluten, dairy, nuts etc. Food dishes refer to a variety of food that is ready to eat such as Thai Green Chicken Curry, Fried Chicken, Butter Bean Salad and so fourth. Participants were asked to circle the rating that best reflected their opinion based on using Likert Scale ratings as seen in Figure \ref{fig:survey}.

\begin{figure}
\centering
\includegraphics[scale=0.8]{images/survey_preferences.png}
\caption{Participants in the survey were asked to rate their food preferences based on using a Likert Scale rating system that best reflected how they felt about the food preference shown. The figure above illustrates a 4 food preferences that are shown to participants.}
\label{fig:survey}
\end{figure}

\begin{figure}
\centering
\includegraphics[scale=0.5]{images/survey_foods.png}
\caption{Participants in the survey were asked to rate food dishes based on using a Likert Scale rating system that best reflected how they felt about the food dish shown. The figure above illustrates a 3 food dishes that are shown to participants.}
\label{fig:survey_food}
\end{figure}

The survey contained 100 food dishes, 21 food preferences, and 5 food intolerances for participants to rate. The full list can be found in Appendix \ref{appendix:survey}. Any unrated items were classified as Neutral ratings inferring participants had no opinion about the food dish at the current time of rating or that they had not tried the food dish before. The survey was grouped into categories for participants to filter through the food dishes efficiently. All participants were shown the same food dishes but the ordering of the categories were randomised to help mitigate biases such as participants rating only the first items they see and so fourth. All food dishes were captured from food vendors in the facility of Kelburn campus at Victoria University of Wellington. Only the title of the food dishes were shown to participants, other information such as images were explicitly excluded. This was to prevent biases around the imagery, the price of the dish, the food vendor of where the food can be purchased etc, since these factors can influence the way participants rate the dishes (especially if majority of participants are students). For example, a participant could like a food dish such as a Butter Chicken Pie from a specific food vendor, but hate it from another food vendor. The image of a food dish may also influence the way participants rate a dish, therefore, excluding this information enables the representation of the participants general food preferences to be captured. A variety of food dishes have also been included in the survey to ensure there are options for all users such as users with intolerances, and users that are vegans or vegetarians. 

Despite this, mistakes were made in the creation of the survey. A small number of food dishes have been incorrectly labelled ``vegetarian" such as Chicken Curry \& Feta, and Chicken Fried Rice. The survey ``Intolerances" section should have also contained the following options: Other, N/A. In addition, the survey should have asked users to only rate food dishes they have explicitly tried before, rather than rating food dishes that best reflected their opinion. The latter is ambiguous since participants may rate food dishes based on dishes they want to try. This is not an accurate representation of actual food preference from the users, which may influence results in the experiment. Another mistake was raised by a vegetarian participant who had notified on the survey that their were no vegetarian dishes in the sushi category. This mistake was based on the choice of food dishes in the survey, since there are 100 food dishes, the length of the survey may cause fatigue in participants and it can be difficult defining a set length.   

The following section details the data collected from the survey. 

\section{Dataset} \label{dataset}

In the context of this project, a rating is referred to as a set of $(u, i, r)$ triplets, where a user $u$ has given a rating $r$ on an item $i$. 
The ratings collected from the survey consist of the following rating values: Hate (-1.0), Dislike (-0.5), Neutral (0.0), Like (0.5), and Love (1.0). Since Binary rating data is only used in the recommender system, Like and Love ratings are considered to be both positive ratings and can be considered as one event, in this case, a Like rating (Referring to any positive rating). Dislike and Hate ratings are also considered as one event and are combined to represent a Dislike event (Referring to any negative rating). Although this project focuses on Binary rating data only, the rationale of collecting multiple rating events was for backup purposes or future evaluation, if binary rating events are found to provide insufficient information to the recommender system. 

The original data collected from the survey consists of the following:
The dataset can be seen in the following table \ref{table:original_dataset}:

\begin{table}[h!]
\centering
\begin{tabular}{|l|l|} 
 \hline
 \multicolumn{2}{|c|}{Original Dataset} \\
     \hline\hline
     Name & Number\\ [0.5ex] 
     \hline
     Users & 91 \\
     \hline
     Food Dishes & 100 \\
     \hline
     Available Food Preferences & 21 \\ 
     \hline
     Available Food Intolerances & 5 \\ 
     \hline
     Ratings & 9100 \\
     \hline
     Like Ratings (Positive) & 4207 \\
     \hline
     Dislike Ratings (Negative) & 1638 \\ [1ex] 
     \hline
     Food Preference Ratings (Negative) & 813 \\ [1ex]
     \hline
\end{tabular}
\caption{A table showing the original dataset collected from the survey.}
\label{table:original_dataset}
\end{table}

% 
% The sparsity of a rating dataset is defined as the density of the vacancies in
% the user-item rating matrix, as in equation 2.1.
% sparsity = 1 −
% # ratings
% # users × # items
% (2.1)
% Because most users would have only rated a very small portion of the
% items, most recommendation datasets have very high sparsity, with sparsity
% as high as 0.95 considered normal [129].3 For this reason, in the field of
% recommender systems, the term “sparse” only refers to the extreme cases
% where the sparsity of the dataset is higher than 0.99. Such extreme sparsity normally occurs when new recommender systems are first established, or
% in datasets where the number of users is small relative to the volume of
% information in the system due to either a large number of items or regular
% updates of the item pool or both.

\subsection{Data Cleansing}

Collaborative filtering algorithms depend heavily on user data to give personalised recommendations. Therefore, the validity of user data can influence how collaborative filtering algorithms perform on the data. To mitigate this, data cleansing is conducted, removing any information from users that provided dubious answers found in the survey. Many answers from the survey contained conflicting ratings from participants, examples were participants who indicated they were vegetarians, but had rated meat dishes positively. These users along with their rating information were removed and were apparent since authentic vegetarian participants rated ``Hate" for every food dish containing meat. A subset of users provided answers to the survey that were difficult to decipher, for example, participants indicated intolerances to gluten but had then positively rated food dishes containing gluten. This could mean several participants may have differing levels of tolerance to gluten, or perhaps expressed they enjoy the gluten free version of the food dish and so fourth. Therefore, any user data that was ambiguous was not removed from the dataset to maintain validity. In addition, cases where participants did not finish rating all items were preserved. Table \ref{table:cleansed_dataset} demonstrates the data after it had been cleansed:

\begin{table}[h!]
\centering
\begin{tabular}{|l|l|} 
 \hline
 \multicolumn{2}{|c|}{Cleansed Dataset} \\
     \hline\hline
     Name & Number\\ [0.5ex] 
     \hline
     Users & 80 \\
     \hline
     Food Dishes & 100 \\
     \hline
     Available Food Preferences & 21 \\ 
     \hline
     Available Food Intolerances & 5 \\ 
     \hline
     Total Ratings & 8000 \\
     \hline
     Like Ratings (Positive) & 3937 \\
     \hline
     Dislike Ratings (Negative) & 1440 \\ [1ex] 
     \hline
     Neutral Ratings (Default) & 2620 \\ [1ex] 
     \hline
     Food Preference Ratings & 613 \\ [1ex] 
     \hline
\end{tabular}
\caption{A table showing the cleansed dataset where all the users that provided contradictory or invalid answers to the questions, had been removed.}
\label{table:cleansed_dataset}
\end{table}

\section{Experimental Background}

In this experiment, an offline evaluation is performed to evaluate the performance of CF algorithms based on the goal of ``Find Good Items". The method used consists of partitioning the dataset, tuning parameters from the recommendation system, and evaluating the performance of CF algorithms using binary recommendation accuracy metrics \cite{zhang}. 

\subsubsection{Data Partitioning} \label{partitioning}

Standard offline evaluations consist of partitioning a dataset into a training set and a test set, containing rating triplets described in Section \ref{dataset}. The training set is used by the recommendation system to learn the food preferences of users, providing recommendations based on predictions made about what the user likes from learning their preferences. Since the training set and test set are separate datasets, all rating triplets in the test set are unseen to the recommendation system and have not been learnt by the recommendation system during the training process. Therefore the ratings inside the test set is used to evaluate the performance of the recommendation system by checking whether the user has liked or disliked items recommended by the system. In addition, this is used to evaluate the generalisability of the recommendation system, giving insight to how well it may perform in a real scenario since the model has not seen the instances from the test data. 

In cases where the recommendation system contains tunable parameters, the dataset is partitioned into three datasets: a training set, a testing set, and a validation set. In the context of recommender systems, the validation set is used to find the optimal parameter settings but can also be used to reduce overfitting of the data, providing insight to the generalisability of the model. The validation set is used to evaluate the performance of the parameters, after the model has learnt on the training set. After the optimal parameters are found from the validation set, the test set is then used to evaluate the final performance of the recommender system. This prevents biases since the optimal parameters have not been learnt using the test set. Partitioning protocols and techniques exist to determine the partitioning of the datasets. This project uses the \textit{skip-every-nth protocol} which is specifically designed for recommender systems \cite{zhang}.

The \textit{skip every nth protocol} \cite{zhang} is used to partition the dataset into a training set and a test set. The \textit{skip-every-nth protocol} consists of randomising the ordering of users, and randomising the ratings within each user in a sequence that is contiguous to the user, aggregating the result in a list sequence \cite{zhang}. The protocol iterates through this list of user ratings assigning every \textit{nth} rating to the test set, the remaining ratings are assigned to the training set. The \textit{skip-every-nth protocol} ensures that ``the sparsity of the training dataset is minimally disturbed, and guarantees a (n - 1):1 size ratio between training and test data for all users up to decimal rounding" \cite{zhang}. However, since the \textit{nth} value affects the user ratings that are involved in the test set, this means users with few ratings may not be in the test set since their ratings may be skipped during the partitioning process. In these cases, evaluation is done \textit{n} times with the same ordering, but offsetting the partitioning each run by 1. This ensures that every user rating is used exactly once in the test set \cite{zhang}. 

The experiments in this report use the \textit{skip-every-10th protocol} to the dataset into a training set and the test set for final evaluation of the recommender system.

\subsubsection{Parameter Tuning}

Standard CF latent factor models have several parameters influencing the accuracy of recommendations, therefore, tuning parameters should be performed for optimal performance. Parameter tuning is typically performed using a training set and a validation set previously described in Section \ref{partitioning}. However, a validation set is not required as K-fold cross validation can be performed to find the optimal parameters of a model instead. 

K-fold cross validation \cite{kfold, campochiaro2009metrics} consists of partitioning a dataset into $k$ equal size folds (subsets). In the context of this project, the rating data contained within each fold is selected randomly. One fold from $k$ folds is selected as the validation set, the remaining folds $(k-1)$ are used as the training set. Evaluation metrics described in Section \ref{accuracy}, Section \ref{precision}, Section \ref{roc}, and Section \ref{auc} are then calculated. This is repeated $k$ times where each fold is used exactly once as a validation set. The evaluation results are then averaged, providing insight to how well the parameters perform for that run. In this project, K-fold validation is repeated multiple times, tuning the parameters in each run to find the optimal parameters. The optimal parameters found from multiple runs are then used for the final evaluation of the recommender system using the test set.

The advantage of K-fold cross validation is that every rating point is used exactly once in the validation set, and used in a training set $k-1$ times \cite{kfold}. However, since K-fold cross validation performs evaluation $k$ times, it can be computationally expensive to run \cite{kfold}.

A standard 10-fold cross validation is used to determine the optimal parameters in our experiments. 10-fold cross validation is performed on the training set, and is not performed on the test set. In this way, the test set remains uncontaminated since the optimised parameters have not been learnt from any rating data from the test set, preventing bias.

\subsubsection{Recommendation Accuracy} \label{accuracy}

In the context of recommender systems, recommendation accuracy measures the performance of a recommender system based on how well the system can correctly distinguish whether a user will like an item or whether they will not. This is similar to the decision users make in a real world scenario where decisions are based on ``acceptance or rejection" of items from the recommendations given \cite{zhang}. Therefore, recommendation accuracy metrics are appropriate for the use case of ``Find Good Items" \cite{evaluation} as it can ``evaluate how well the system is at assisting the final decision of the user" \cite{zhang}. 

Since the goal of this project is to ``Find Good Items", recommendation accuracy is used as a primary metric in our experiments. Although recommendation accuracy can measure the good and bad items for each user, it can not measure how much a user will like a food dish, or how much a user will dislike a food dish. All recommended food dishes are predicted to be liked by the user, and all unrecommended food dishes are predicted to be disliked. This means recommendation accuracy is unable to determine the ranking of preference for recommendations given to users, not being guaranteed to see their top preference in the recommendations list based on $n$ items. 

For these reasons, recommendation accuracy is used with supporting metrics of precision (Section \ref{precision}), recall (Section \ref{precision}), precision at top $N$ (Section \ref{precision}), and the area under the receiving operating characteristic curve (Section \ref{auc}). 

\subsubsection{Precision, Recall, and Precision at N} \label{precision}

There are four possible categories that binary recommendations from a recommender system can be classified as \cite{zhang}. These four possible categories can be seen in Table \ref{table:categories}. Recommended items preferred by users result in a True-Positive (TP). This represents an accurate result from the recommender system, since the user preferred the recommended item. Recommended items not preferred by users result in a False-Positive (FP). This represents an inaccurate result from the recommender system, since the user did not prefer the recommended item. Non-recommended items preferred by users result in a False-Negative (FN). This represents an inaccurate result from the recommender system, since the user preferred an item, but the item was not recommended by the recommender system. Non-recommended items not preferred by users result in a True-Negative (TN). This represents an accurate result from the recommender system, since the user did not prefer an item, and the item was not recommended by the recommender system.

\begin{table}[h!]
\centering
\begin{tabular}{|l|l|l|} 
     \hline
      & Recommended & Not Recommended \\ [0.5ex] 
     \hline
     Preferred & True-Positive (TP) & False-Negative (FN)\\
     \hline
     Not Preferred & False-Positive (FP) & True-Negative (TN) \\
     \hline
\end{tabular}
\caption{Categories recommendations}
\label{table:categories}
\end{table}

For binary ratings, precision and recall are often used as metrics. Precision measures the relevant recommendations (fidelity) to the user shown in Equation \ref{eq:precision}, whereas recall measures the relevant recommendations from the total preferred items (completeness) \ref{eq:precision}. 

Precision at $N$ uses Equation \ref{eq:precision} but only looks at the top $N$ recommendations list where $N$ is the number of recommended top items to the user. Precision at $N$ gives insight as to how many true liked items are in that list, and how many false liked items are not in the list. This can be used to measure the portion of relevant items that are recommended to users based on the top $N$ recommendation list, where only $N$ recommendations are shown.

\begin{equation} \label{eq:precision}
\centering
Precision = \frac{\#TP}{\#TP + \#FP}
\end{equation}

\begin{equation} \label{eq:recall}
\centering
Recall = \frac{\#TP}{\#TP + \#FN}
\end{equation}

% Recall, in its purest sense, is almost always impractical to measure in a
% recommender system. In the pure sense, measuring recall requires knowing
% whether each item is relevant; for a movie recommender, this would involve
% asking many users to view all 5000 movies to measure how successfully we recommend
% each one to each user

% Perhaps a more appropriate way to approximate precision and recall would
% be to predict the top N items for which we have ratings. That is, we take a
% user’s ratings, split them into a training set and a test set, train the algorithm
% on the training set, then predict the top N items from that user’s test set. If we
% assume that the distribution of relevant items and nonrelevant items within
% the user’s test set is the same as the true distribution for the user across all
% items, then the precision and recall will be much closer approximations of the
% true precision and recall. This approach is taken in Basu et al. [1998].
% In information retrieval, precision and recall can be linked to probabilities
% that directly affect the user. If an algorithm has a measured precision of 70%,
% then the user can expect that, on average, 7 out of every 10 documents returned
% to the user will be relevant. Users can more intuitively comprehend the meaning
% of a 10\% difference in precision than they can a 0.5-point difference in mean
% absolute error

% One of the primary challenges to using precision and recall to compare different
% algorithms is that precision and recall must be considered together to
% evaluate completely the performance of an algorithm.

% Precision alone at a single search length or a single recall level can be appropriate
% if the user does not need a complete list of all potentially relevant
% items, such as in the Find Good Items task. If the task is to find all relevant
% items in an area, then recall becomes important as well. However, the search
% length at which precision is measured should be appropriate for the user task
% and content domain.

\subsubsection{ROC Curves and Area Under The Curve} \label{roc} \label{auc}
In the context of this project, Receiver Operating Characteristic (ROC) curves can be used to visualise all the trade-offs between the True-Positive Rate, and the False-Positive Rate of a recommender system. The ROC cruve is computed by using all possible thresholds on a graph , and visualising the trade-off between the True-Positive Rate and the False-Positive Rate, displaying how well a model is able to distinguish recommendations that the users find relevant. The True Positive (TP) rate which is the y-axis, shows the sensitivity of the model. This means that it shows the fraction in which liked items by specific users are correctly recommended to the users. The False Positive (FP) rate which is the x-axis, shows the fraction in which disliked items by specific users are incorrectly recommended as items they would like. The Area Under The Curve (AUC) is a measure that can determine how good the recommender system is over all the thresholds, where a AUC of 1.0 specifies a recommendation system that recommends items perfectly to users. 

\section{Experimental Design}

\subsection{Setup/Tools}
The following equipment and tools were used to run the experiment:
\begin{itemize}
	\item{Intel(R) Core(TM) i7-4770 CPU @ 3.40GHz}
	\item{NVIDIA Quadro K620 GPU 2GB DDR3}
	\item{8GB DD3 RAM}
	\item{Arch 64bit GNU/Linux Operating System}
	\item{The programming language Ruby for the application WOTM}
	\item{PredictionIO 9.4 using HBase for the event server, and running on Apache Spark}
	\item{PredictionIO Engine written in Scala using Apache MLLib}
\end{itemize}

\subsection{Method} \label{method}

\begin{figure}
\centering
\includegraphics[scale=0.7]{recent_images/evaluation_workflow.png}
\caption{Evaluation for 1 run. Multiple runs are computed, and the results are averaged.}
\label{fig:evaluation_workflow}
\end{figure}

An offline study is evaluated using the dataset collected from the user study \ref{fig:evaluation_workflow}. The evaluation process can be shown in Figure \ref{fig:evaluation_workflow}. \textit{Skip-every-10th protocol} is used as the partitioning strategy to create a training set and a test set. The test set is used for the final evaluation of the learnt model when the optimal parameters have been found. 10-fold cross validation is performed on the training set multiple times with different parameters for the model until the optimal parameters are discovered. Since cross-validation uses each subset as validation data, we retrain the model on the whole training set data with the optimal parameters obtained from the k-fold validation. Using this model, evaluation is done on the test set computing the recommendation accuracy, the precision, the recall, and the precision at N (10). The results are then averaged over 30 runs. This method means that the parameters are not finalized since different optimal parameters are found every run. 

Therefore, another method is used to finalize the parameters. This method consists of doing the same above, but instead of doing the optimal parameters for every run, we use the same optimal parameters from the first run to test the remaining 29 runs. In this way, the parameters are finalized but the results of the data are contaminated since the learnt model in the remaining runs may have been trained on data from the test dataset from the first run.  

\section{Experimental Results}

In this section we present the results of the experiments described in Section \ref{method}. We will focus on comparing the hybrid CF approach to the Single SVD CF approach, demonstrating the use of food dish content used inside the hybrid CF technique is able to outperform the Single CF SVD approach. 


% This section also demonstrates the optimal parameters for the different approaches for further investigation in a commercial/real world environment.  

\begin{figure}[!tbp]
  \centering
  \begin{minipage}[b]{0.45\textwidth}
    \includegraphics[width=\textwidth]{recent_images/accuracy.png}
    \caption{Average accuracy from 30 runs for different algorithms.}
    \label{fig:accuracy}
  \end{minipage}
  \hfill
  \begin{minipage}[b]{0.45\textwidth}
    \includegraphics[width=\textwidth]{recent_images/precision.png}
    \caption{Average precision from 30 runs for different algorithms}
    \label{fig:precision}
  \end{minipage}
\centering
  \hfill
  \begin{minipage}[b]{0.45\textwidth}
    \includegraphics[width=\textwidth]{recent_images/recall.png}
    \caption{Average recall from 30 runs for different algorithms}
    \label{fig:recall}
  \end{minipage}
  \hfill
  \begin{minipage}[b]{0.45\textwidth}
    \includegraphics[width=\textwidth]{recent_images/auc.png}
    \caption{Average area under the roc curve for 30 runs over different algorithms}
    \label{fig:auc}
  \end{minipage}
\end{figure}

\begin{figure}
\centering
\includegraphics[scale=0.6]{recent_images/top_n.png}
\caption{Average precision at top n (10) for 30 runs over different algorithms}
\label{fig:top_n}
\end{figure}

\begin{figure}
\centering
\includegraphics[scale=0.3]{images/results_tuning.png}
\caption{Averaged results ran 30 times with the same optimal parameters (contaminated).}
\label{fig:results}
\end{figure}

\subsection{Average Recommendation Accuracy}
Results are shown in \ref{fig:results} and \ref{fig:accuracy}. Recommendation accuracy is the most suitable for the use case of ``Find Good Items". From the results, the hybrid collaborative filtering seems to provide the highest accuracy in this case. The preferences from each user describing the attributes the user likes and dislikes, provides extra information to the recommender system, thus resulting in the higher accuracy. However, this information may not be available. The Hybrid CF approach was able to achieve an accuracy an 87.06 \%, where as the Dual SVD system achieved an 86.74\% accuracy. Although this is just a minor difference, the attributes of a food dish only contain three categories: meat type, food type, and cuisine type. These attributes provide extra information about the user, therefore provide higher accuracy. Although tagging of this information for each dish can be tedious, and is expected that errors may occur in labelling the data which may have affected the accuracy. In the other systems, Single SVD with an accuracy of 82\% was able to outperform the popularity baseline which had a accuracy of 73.84\%. This is expected since Single SVD is able to provide personalised recommendations whereas non-personalised are given by the popularity baseline. An unexpected case was the Item SVD was able to achieve an accuracy of 82.64\%, similar to that of the Single SVD 82.43\%. 

\subsection{Average Precision}

 From the results shown in \ref{fig:precision} and \ref{fig:results}, Dual SVD was able to achieve a higher precision (93.35\%) over the Single SVD (92.56\%) approach over multiple runs (30). This means the Dual SVD approach was able to more accurately recommend items the user preferred rather than the Single SVD. Since the average precision of the Single SVD is lower, it means that the recommender system recommended more items that users indicated they disliked (false positives). The lowest precision was the popularity baseline which provided non-personalised recommendations with a precision of 90\%. Item SVD achieved 91.64\% precision, where as the Hybrid CF approach achieved the highest precision of 94.19\%. 

\subsection{Average Recall}

From the results shown in \ref{fig:recall} and \ref{fig:results}, the system with the highest recall was the Dual SVD (89.03\%). The popularity baseline had the lowest recall of (72.21\%). The other system had the following recall rates: Single SVD (84.74\%), Item SVD (85.60\%), and Hybrid CF (87.06\%). Interestingly enough, the Hybrid CF performed better on all the other previous metrics (Accuracy and Precision), but had a lower recall rate than Dual SVD. Recall is closely related to Precision and their is usually a tradeoff between the two, since recall measures the amount of good items recommended from the total preference list of the user. This meant the Hybrid CF approach had recommended less of the total preferred items of users, than the Dual SVD approach. This might be the reason why the Hybrid approach had performed better than the Dual SVD in other metrics, since the sensitivity of the recommendations were higher than the Dual SVD system.

\subsection{Average AUC}

From the results shown in \ref{fig:auc} and \ref{fig:results}, the Hybrid CF approach achieved the highest AUC with 91.50\%. The Dual SVD achieved a similar result of 91.20\%. These results imply that the recommendations are good, since 90\% represents a good overall recommendation classifier holistically. Item SVD achieved a AUC of 86.67\%, Single SVD achieved a AUC of 84.76\%, and the popularity baseline achieved an AUC of 72.21\%. These indicate that the other systems performed fairly well. 

\subsection{Average Precision at top N (10)}

From the results shown in \ref{fig:top_n} and \ref{fig:results} all the systems have the same average precision of 27.08\% when 10 items are recommended to them. This means the same amount of preferred user items have appeared in the top 10 recommendations list. This is metric is suited towards providing insight about how users may perceive the recommendation algorithm in a real-world environment. For example, in an offline evaluation, users have already indicated the food dishes they have liked. If these food dishes are displayed at the top of the recommendations list, then it may establish trust between the user and the recommender system. 

\subsection{Experimental Conclusions}

In terms of the case of ``Find Good Items", all systems were able to achieve the goal, as they all provided personalised recommendations that performed better than non-personalised recommendations based on item popularity. In particular, the Hybrid CF approach was able to provide personalised recommendations at an accuracy of 87.06\%, outperforming the popularity baseline which provided non-personalised recommendations at an accuracy of 73.84\%. This fulfills the objectives 1 and 2 in \ref{experiment_objectives}. 

In terms of objective 3 in \ref{experiment_objectives}, the Item SVD achieved worse performance in all metrics compared to the Hybrid CF approach except precision at top n (10), where it achieved the same result. Item SVD achieved an accuracy of 82.65\%, being dominated by the Hybrid CF approach of 87.06\%. 

The Dual SVD was able to achieve an accuracy of 86.74\%, outperforming Single SVD which achieved 82.43\%. In addition, Dual SVD achieved better results in all other metrics except precision at top N (10), where the result was the same. Therefore, correlated recommendations based on binary ratings provide more accurate recommendations than non-correlated binary ratings. This claim backs up the claim that taking into the relationship between Like/Dislike of items provides overall better recommendations. 

Overall, a Hybrid CF approach was able to outperform every other system in terms of Accuracy, Precision, Area Under the Curve (ROC). Although the Dual SVD achieved a higher recall than the Hybrid, suggesting that more ``Good" items were recommended by the Dual SVD, the Dual SVD may not be better than the Hybrid CF approach since it recommended more items that users did not prefer. Ultimately, this is an important factor to consider, since recommending items users may not like, may cause distrust in the recommendations, leading to users less likely to use the system. 

\section{Limitations to Generalisability}

An offline experiment may not be enough as users may feel different in a real user evaluation. Therefore the following points indicate limitations to generalisability:
\begin{itemize}
	\item{Contradictions in the data and participation may lead to inaccurate recommendations}
	\item{Proportion of items and user may not be accurate in a real system}
	\item{Density and sparsity of the user-item rating matrix was densely filled.}
	\item{The survey questions need to ask the right questions}
\end{itemize}
Some people may have filled out dishes that they haven't tried before. This could affect the results because often what people do is different to what they think. 
There were cases where people filled out the form inaccurately, such that they chose random dishes. We know this because there are cases where someone states that they are “vegetarians” but then say they like “meat” etc.
Cleanse these cases to give better predictions

In addition, recommendation accuracy is identifying items that the user is already aware of, which makes it susceptible to biases involving overfitting and non-novel recommendations \cite{evaluation}. 




\chapter{Conclusions}\label{C:con}

\section{Contributions}
In this project, the following contributions have been made. 
\begin{enumerate}
	\item{A user study has been conducted collecting a dataset containing 91 users, 100 food items, 4207 positive ratings, and 1638 negative ratings.}
	\item{A Single SVD CF System, Dual SVD CF System, Item CF System, and a Hybrid CF System have been implemented based on utilizing Binary Ratings (Likes/Dislikes).}
	\item{An offline evaluation has been conducted for each of the systems above}
	\item{Integration of recommender system has been implemented in WOTM}
	\item{Demonstration that a Hybrid CF approach was able to provide personalised recommendations at an accuracy of 87.06\%, outperforming the popularity baseline which provided non-personalised recommendations at an accuracy of 73.84\%}
\end{enumerate}
\section{Future Work}

\subsection{User Evaluation or Online Evaluation}
In this experiment, we have demonstrated that a Hybrid CF approach performs best in an offline study. The next step is to conduct an online or user evaluation to see how users respond to the recommendations of each algorithm. From this, we are able to truly see how each algorithm performs in a real scenario, as well as test the novelty and serendipity factor and so forth. 

\subsection{Content-Based Filtering}
The content-based filtering approach used in the experiments is a simple one used with the Hybrid SVD System. Currently each dish is categorised by three attributes and is manually labelled. In future work, depending on the attributes of the each food dish, we may want finer granularity, therefore, giving more precise content-based filtering recommendations based on preferences specified by users.  In future work, the content-based filtering approach may want to be treated as a classification problem, where machine learning algorithms will automatically label dishes based upon the contents of the dish and automatically learn what the user likes based on previous rating patterns. This can be used instead of the current approach where users explicitly label their preferences, which may interrupt the user experience of the application WOTM. In addition, pure content-based filtering approaches may want to be added into the hybrid collaborative filtering approach for the benefit of increasing the recommendation accuracy and relieving the ``Cold Start" problem CF suffers from. Although a hybrid CF approach can alleviate the ``Cold Start" problem by using content-based filtering until there are enough ratings for accurate recommendations, the factors of complexity, scalability, and efficiency, must be considered when examining content-based filtering techniques. For example, algorithms such as Naive Bayes may be a good starting option since naive Bayes is efficient and generally gives good results, however, naive Bayes may need to run in real time to compute similar products when a user indicates they have liked a food dish, affecting the efficiency. These trade-offs have to be considered as each one may affect the user experience of the WOTM application. 

\subsection{Serendipity, Novelty, and Trust}
Serendipity and novelty are difficult to measure in offline evaluations. Trust is another factor difficult to measure in offline studies. In these cases, online evaluation or user studies may need to be conducted. Serendipity and Novelty are based on a form of randomness. This is used to help steer recommendations toward new items or serendipitous items. For example, in a KNN user based CF approach, serendipity or novelty may be adjusted depending on how many K neighbours (users) to take into account for.Considering a small number K neighbours, may lead to obvious recommendations for the current user, whereas choosing a higher number of K neighbours may lead to novel or serendipitous recommendations. For future work, perhaps the SVD algorithm can be combined with a neighbourhood approach to provide some form of serendipity. Another approach that may help the hybrid approach in this paper, could be the adjustments of boost from the content-based filtering on the collaborative filtering approach. Higher boosted content values may lead to more obvious choices for the active user, whereas, lower approaches, would be similar in finding more serendipitous dishes based on other users ratings (CF).  

Trust also should be considered when factoring in novelty and serendipity into the recommendation engine. New users to the application of WOTM, may test the recommendation system to establish trust before they start using the application regularly. In future work, this should be taken into account, and should also be tested in online studies. For example, a recommendation engine that tries to provide new users with novel or serendipitous food dishes may invoke less trust and confidence for the new users. This may eventually lead to new users not trusting the system, and abandoning the application, or disregarding any of the personalised recommendations given. In order to combat this, future work should provide the most obvious choices to the users (perhaps by boosting the content), and then as time progresses, start introducing novel and serendipitous recommendations to the users. In addition, the application may allow users to set the settings for these users, enabling user specific goals to be accomplished. 

\subsection{Contextual events}
In future work, context events provide more information to the recommendation engine in terms of the device the user is using to access the application, and the location of where the active user is. Another way that may make the recommendation more accurate is to use user search content. This would require adding a search engine which would then index all the users searches to provide a boost of recommendations based on the content or food dish searched for. This context information should boost the strength of the collaborative filtering search, and also have a decaying factor since the context of the events have purpose behind it. For example, on a hot day, a user may want ice cream and search for it. But on the next day, they may want feel like something new. For these reasons, a decaying factor should be added into the engine.

\section{Conclusion}

This project has explored recommendation techniques and has integrated a recommender system into the web-centric mobile application What's On The Menu (WOTM). A user study has been conducted collecting a dataset containing 91 users, 100 food items, 4207 positive ratings, and 1638 negative ratings.

A baseline predictor system recommending non-personalised items based on popularity was compared to four recommender systems, providing personalised recommendations. These systems were an Item-based SVD approach, a Single SVD approach, a Dual SVD approach, and a Hybrid CF approach. This recommendation system provided personalised recommendations to users based on the use case of ``Find Good Items". 

The Hybrid CF system was able to outperform every other system in terms of Accuracy, Precision, Area Under the Curve (ROC). In particular, the Hybrid CF approach was able to provide personalised recommendations at an accuracy of 87.06\%, outperforming the popularity baseline which provided non-personalised recommendations at an accuracy of 73.84\%, fulfilling the goal of ``Find Good Items".  

For further future work, an online study should be examined to see whether the Hybrid approach is able to provide similar results in a real world environment.
\begin{appendices}

\chapter{\todo{Result Tables from thing}} \label{appendix:results_table}

\begin{figure}
\centering
\includegraphics[scale=0.3]{appendices/single_als_30_runs.png}
\caption{Individual 30 runs of the standard single ALS algorithm. (contaminated)}
\label{fig:single_algorithm}
\end{figure}

\begin{figure}
\centering
\includegraphics[scale=0.3]{appendices/dual_als_30_runs.png}
\caption{Individual 30 runs of the dual ALS algorithm. (contaminated)}
\label{fig:dual_algorithm}
\end{figure}

\begin{figure}
\centering
\includegraphics[scale=0.3]{appendices/hybrid_als_30_runs.png}
\caption{Individual 30 runs of the Hybrid ALS algorithm which takes into account the content preferences of the user and considers these preferences when making recommendations. (contaminated)}
\label{fig:dual_algorithm}
\end{figure}

\chapter{Survey} \label{appendix:survey}

The contents...

\includepdf[pages={1-2}]{appendices/survey_1.pdf}
\includepdf[pages={1-7}]{appendices/survey_2.pdf}

\chapter{Human Ethics Committe (HEC) Approval} \label{appendix:hec}

\begin{figure} 
\centering
\includegraphics[scale=0.3]{appendices/hec_approval.png}
\caption{Human Ethics Approval for the experiment.}
\label{fig:hec_approval}
\end{figure}

\includepdf[pages={1-2}]{appendices/participant_form.pdf}

\end{appendices}



%%%%%%%%%%%%%%%%%%%%%%%%%%%%%%%%%%%%%%%%%%%%%%%%%%%%%%%

\backmatter

%%%%%%%%%%%%%%%%%%%%%%%%%%%%%%%%%%%%%%%%%%%%%%%%%%%%%%%


% % \bibliographystyle{ieeetr}
% \bibliographystyle{acm}

\bibliography{sample}
\bibliographystyle{IEEEtranN/IEEEtranN}

% \nocite{*}

\end{document}
