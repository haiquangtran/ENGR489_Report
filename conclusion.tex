\chapter{Conclusions}\label{C:con}
The conclusions are presented in this Chapter.

\section{Contributions}
\begin{itemize}
	\item{Integrate CF with WOTM}
	\item{Develop three various CF}
	\item{Create dataset}
	\item{Evaluate and compare}
	\item{Percentage etc}
	\item{item popularity}
\end{itemize}

\section{Future Work}

\subsection{User Evaluation or Online Evaluation}
In this experiment, we have identified the optimal parameters for each algorithm in order to do an online or user evaluation study. The next step is to conduct an online or user evaluation to see how users respond to the recommendations of each algorithm. From this, we are able to truly see how each algorithm performs in a real scenario, as well as test the novelty and serendipity factor and so forth. 

\subsection{Content-Based Filtering}
The content-based filtering approach we use with ALS is simple, yet effective. Currently each dish is categorised by three attributes and is manually labelled. In future work, depending on the attributes of the each food dish, we may want finer granularity, therefore, giving more precise content-based filtering recommendations based on preferences specified by users.  In future work, the content-based filtering approach may want to be treated as a classification problem, where machine learning algorithms will automatically label dishes based upon the contents of the dish and automatically learn what the user likes based on previous rating patterns. This can be used instead of the current approach where users explicitly label their preferences, which may interrupt the user experience of the application WOTM. In addition, pure content-based filtering approaches may want to be added into the hybrid collaborative filtering approach for the benefit of increasing the recommendation accuracy and relieving the ``Cold Start" problem CF suffers from. Although a hybrid CF approach can alleviate the cold start problem by using content-based filtering until there are enough ratings for accurate recommendations, the factors of complexity, scalability, and efficiency, must be considered when examining content-based filtering techniques. For example, algorithms such as Naive Bayes may be a good starting option since naive Bayes is efficient and generally gives good results, however, naive Bayes may need to run in real time to compute similar products when a user indicates they have liked a food dish, affecting the efficiency. These trade-offs have to be considered as each one may affect the user experience of the WOTM application. 

\subsection{Serendipity, Novelty, and Trust}
As mentioned earlier in Section \todo{do this}, serendipity and novelty are difficult to measure in offline evaluations. Trust is another factor difficult to measure in offline studies. In these cases, online evaluation or user studies may need to be conducted \todo{Check this}. Serendipity and Novelty are based on a form of randomness. This is used to help steer recommendations toward new items or serendipitous items. For example, in a KNN user based CF approach, serendipity or novelty may be adjusted depending on how many K neighbours (users) to take into account for. \todo{Considering a small number K neighbours, may lead to obvious recommendations for the current user, whereas choosing a higher number of K neighbours may lead to novel or serendipitous recommendations}. For future work, perhaps the ALS algorithm can be combined with a neighbourhood approach to provide some form of serendipity. Another approach that may help the hybrid approach in this paper, could be the adjustments of boost from the content-based filtering on the collaborative filtering approach. Higher boosted content values may lead to more obvious choices for the active user, whereas, lower approaches, would be similar in finding more serendipitous dishes based on other users ratings (CF).  

Trust also should be considered when factoring in novelty and serendipity into the recommendation engine. New users to the application of WOTM, may test the recommendation system to establish trust before they start using the application regularly. In future work, this should be taken into account, and should also be tested in online studies. For example, a recommendation engine that tries to provide new users with novel or serendipitous food dishes may invoke less trust and confidence for the new users. This may eventually lead to new users not trusting the system, and abandoning the application, or disregarding any of the personalised recommendations given. In order to combat this, future work should provide the most obvious choices to the users (perhaps by boosting the content), and then as time progresses, start introducing novel and serendipitous recommendations to the users. In addition, the application may allow users to set the settings for these users, enabling user specific goals to be accomplished. 

\subsection{Contextual events}
In future work, context events provide more information to the recommendation engine in terms of the device the user is using to access the application, and the location of where the active user is. Another way that may make the recommendation more accurate is to use user search content. This would require adding a search engine which would then index all the users searches to provide a boost of recommendations based on the content or food dish searched for. This context information should boost the strength of the collaborative filtering search, and also have a decaying factor since the context of the events have purpose behind it. For example, on a hot day, a user may want ice cream and search for it. But on the next day, they may want feel like something new. For these reasons, a decaying factor should be added into the engine.

\section{Conclusion}